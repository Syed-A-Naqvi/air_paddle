\documentclass{article}
\usepackage[utf8]{inputenc}
\usepackage{amsmath, amssymb}
\usepackage[top=1in, bottom=1in, left=1in, right=1in]{geometry}

\title{\textbf{AirPaddle: Vision-Based Interactive Ball Simulation}\\ \LARGE{CSCI 4220U \& CSCI 3010U}}
\author{Syed Arham Naqvi \\ 100590852 \\ syedm.naqvi@ontariotechu.net}
\date{\today}

\begin{document}
\maketitle

\section*{\Large{Summary}}
This project proposes an interactive game combining computer vision techniques with simulation and modeling. The goal is to develop a real-time system where the user's hand position is translated to
in-game platform coordinates using computer vision, and a ball is simulated to bounce around the screen where all but the bottom edge are impenetrable. The user must use their hand to move the platform and prevent the ball
from falling off the screen. The system integrates a physics-based simulation for ball movement and collision dynamics, while computer vision is responsible for detecting and tracking the user's hand in real-time.

\vspace*{5pt}

\section*{\Large{Methodology}}

\begin{center}
    \begin{minipage}[t]{0.48\textwidth}
        \subsection*{\large{Computer Vision}}
        \begin{itemize}
            \item Use OpenCV and a webcam for real-time hand tracking.
            \item Implement contour detection or models like MediaPipe Hands.
            \item Map hand position to control the in-game paddle.
        \end{itemize}
    \end{minipage}
    \begin{minipage}[t]{0.48\textwidth}
        \subsection*{\large{Simulation and Modeling}}
        \begin{itemize}
            \item Simulate ball motion using Newtonian mechanics.
            \item Implement realistic ball-hand and ball-wall collisions.
            \item Assume fully elastic ball physics with no friction.
        \end{itemize}
    \end{minipage}
\end{center}

\begin{center}
    \begin{minipage}[t]{0.96\textwidth}
        \subsection*{\large{Game Mechanics}}
        \begin{itemize}
            \item The ball moves under gravity and bounces realistically (assuming full elasticity).
            \item The user must keep the ball in play using their hand.
            \item Score tracking, game-over conditions and difficulty settings (multiple balls) are implemented.
        \end{itemize}
    \end{minipage}
\end{center}

\vspace*{5pt}

\section*{\Large{Challenges and Solutions}}

\begin{center}
    \begin{minipage}[t]{0.48\textwidth}
        \subsection*{\large{Hand Detection Accuracy}}
        \begin{itemize}
            \item \textbf{Challenge:} Lighting and occlusion can affect tracking.
            \item \textbf{Solution:} Use adaptive thresholding or deep learning models.
        \end{itemize}
    \end{minipage}
    \begin{minipage}[t]{0.48\textwidth}
        \subsection*{\large{Real-Time Performance}}
        \begin{itemize}
            \item \textbf{Challenge:} Maintain smooth gameplay with low latency.
            \item \textbf{Solution:} Utilize deep learning models and GPU processing.
        \end{itemize}
    \end{minipage}
\end{center}

\end{document}
